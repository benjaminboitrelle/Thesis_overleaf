%\chapter*{Introduction\markboth{INTRODUCTION}{INTRODUCTION}} \addcontentsline{toc}{chapter}{\protect\numberline{}Introduction}
\chapter{Introduction}

  In 2012, the \gls{LHC} has detected a new particle compatible with the boson predicted by the Higgs-Englert Brout mechanism, which explains the spontaneous electro-weak symmetry breaking.
  Although the energy and luminosity upgrades could improve the knowledge on this new particle and also find existence of physics beyond the \gls{SM}, the complex environment of the events generated by the \gls{LHC} hide fundamental parameters that help to perform precise measurements.

  One of the biggest scientific projects is being prepared. 
  The \gls{ILC} will be a linear eletron-positron collider with a length of 31 kilometres and a center of mass energy  $\sqrt{s} = 250~-~500~\rm{GeV}$ (with a possible upgrade to $1~\rm{TeV}$). 
  It will be able to perform more accurate measurements of known particles (like the coupling of Higgs boson to the fermions), but also to study the dark matter and physics beyond the \gls{SM}. 
  This project imposes new challenges on the instrumentation side. 
  For example, to measure the Higgs coupling to the charm quarks, a precise measurement of the secondary vertices created close to the interaction point is needed.
  The inner part of the detector dedicated to reconstruct vertices should combine a good spatial resolution ($\leq 3~\rm{\mu m}$) and a material budget of less than a thousand of the radiation length ($\rm{X_0}$).
  This subdetector, called vertex detector, should be optimised to perform tracking in a high density particles environment.
  The PLUME collaboration is developing tools to overcome this challenge thanks to an innovative concept of double-sided pixelated ladders for tracking, called \gls{PLUME}. 
  This detector is equipped with six CMOS pixels sensors, placed next to each other on each side of a very light mechanical structure. 
  The collaboration tries to reach a material budget close to $0.35~\%~\rm{X_0}$. 
  For every track, two positions will be measured, one on each side. 
  This double-measurement will help to determine the intersection point of the particle with the detector, but also to know the origin and the movement of the particles.

  Different aspects of the vertex detector development are discussed in the thesis.
  This work gives an overview of the validation and characterisation of such complicated detector and aims to give its performances, such as the spatial resolution, the benefits of double-sided measurements and the material budget of such device.

  The document is organised as followed:
  Chapter~\ref{chap:SM} presents the theoretical context, with an overview of the \gls{SM} and theories beyond the \gls{SM}.
  Chapter~\ref{chap:ILC} describes the experimental context of the thesis, by describing the future linear collider \gls{ILC} and focusing especially on one of the experiment, the \gls{ILD}.
  Chapter~\ref{chap:phyics} introduced the different physics studies that will be performed at the \gls{ILC} and focus especially on a possible analysis of the $\nu\overline{\nu}H$ channel at the \gls{ILC}.
  In chapter~\ref{chap:vxd}, the different \gls{VXD} for the \gls{ILD} are presented, as well as a description of the \gls{PLUME} collaboration and the status of the detectors produced.
  The three last chapters are devoted to the studies performed during this thesis.
  In chapter~\ref{chap:labTests}, the validation in the laboratory of the different \gls{PLUME} modules are reported.
  Chapter~\ref{chap:deformation} is presenting the observation of the ladder deformation during a test beam campaign which was done in 2011 at CERN. 
  It also shows the benefits of a double-sided measurement compared to a single-sided ladder.
  Chapter~\ref{chap:X0} deals with the measurement of the radiation length of the first fully working \gls{PLUME} ladder, which has a weighted material budget ($\rm{X_0}$) estimated to be $0.65~\%~\rm{X_0}$.
  Finally, the conclusion summarises the work performed during the thesis and the outlook is discussed.
