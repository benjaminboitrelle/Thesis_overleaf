%\chapter*{Introduction\markboth{INTRODUCTION}{INTRODUCTION}} \addcontentsline{toc}{chapter}{\protect\numberline{}Introduction}
\chapter{Introduction}

The thesis is organised as followed:
Chapter~\ref{chap:SM} presents the theoretical context, with an overview of the \gls{SM} and theories beyond the \gls{SM}.
Chapter~\ref{chap:ILC} describes the experimental context of the thesis, by describing the future linear collider \gls{ILC} and focusing especially on one of the experiment, the \gls{ILD}.
Chapter~\ref{chap:phyics} introduced the different physics studies that will be performed at the \gls{ILC} and focus especially on a possible analysis of the $\nu\overline{\nu}H$ channel at the \gls{ILC}.
In chapter~\ref{chap:vxd}, the different \gls{VXD} for the \gls{ILD} are presented, as well as a description of the \gls{PLUME} collaboration and the status of the detectors produced.
The three last chapters are devoted to the studies performed during this thesis.
In chapter~\ref{chap:labTests}, the validation in the laboratory of the different \gls{PLUME} modules are reported.
Chapter~\ref{chap:deformation} is presenting the observation of the ladder deformation during a test beam campaign which was done in 2011 at CERN. 
It also shows the benefits of a double-sided measurement compared to a single-sided ladder.
Chapter~\ref{chap:X0} deals with the measurement of the radiation length of the first fully working \gls{PLUME} ladder, which has a weighted material budget ($\rm{X_0}$) estimated to be $0.65~\%~\rm{X_0}$.
Finally, the conclusion summarises the work performed during the thesis and the outlook is discussed.
