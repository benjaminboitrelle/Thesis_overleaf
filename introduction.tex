\chapter*{Introduction\markboth{INTRODUCTION}{INTRODUCTION}}
\addstarredchapter{Introduction}

  In 2012, the \gls{LHC} detected a new particle compatible with the boson predicted by the Higgs-Englert-Brout mechanism, which explains the spontaneous electro-weak symmetry breaking among elementary interactions.
  Although the energy and luminosity upgrades could improve the knowledge of this new particle and also find existence of physics beyond the \gls{SM}, the complex environment of the events generated by the \gls{LHC} hides fundamental parameters of the collisions that help to perform precise measurements.

  To overcome this limitation and complement the \gls{LHC} programme, one of the biggest scientific projects is under preparation. 
  The \gls{ILC} will be a linear eletron-positron collider with a length of 31 kilometres and a centre-of-mass energy ranging from  $250$ to $500~\rm{GeV}$ (with a possible upgrade to $1~\rm{TeV}$). 
  It will be able to perform more accurate measurements of known particles (like the coupling of the Higgs boson to fermions), but also to study the dark matter and physics beyond the \gls{SM}. 
  
  This project imposes new challenges on the instrumentation side. 
  For instance, to measure the Higgs coupling to charm quarks, a precise measurement of the secondary vertices created close to the interaction point is needed.
  The inner part of the detector used to reconstruct vertices, should combine a good spatial resolution ($\leq 3~\rm{\mu m}$) and a material budget of less than a thousandth of the radiation length ($\rm{X_0}$).
  This subdetector, called the vertex detector, should be optimised (geometry, granularity, timing) to perform tracking in a high particle density environment.
  
  The \gls{PLUME} collaboration is developing devices to overcome this challenge thanks to an innovative concept of double-sided detection layers.
  Two families of prototypes have been built.
  Both exploit the \gls{CMOS} technology for the pixel senors which cover each side of the device layer.
  They differ by their material budget.
  While the first prototype focus on the electric functionality, the second prototype targets a material budget of $0.35~\%$ radiation length ($\rm{X_{0}}$).
  The main purpose of this work is to validate the benefits of \gls{PLUME} concept and characterise the performances of the prototypes in terms of spatial resolution, angular resolution and confirm their actual material budget.
  
  %This detector is equipped with six \gls{CMOS} pixels sensors, placed next to each other on each side of a very lightweight mechanical structure. 
  %The collaboration tries to reach a material budget close to $0.35~\%~\rm{X_0}$. 
  %For each track, two positions will be measured, one on each side. 
  %This double-measurement will help to determine the intersection point of the particle with the detector, but also to estimate the origin and the momentum of the particles.

  This work gives an overview of the validation and characterisation of such a complex detector and aims to detail its performance, such as the spatial resolution, the benefits of double-sided measurements and confirmed the actual the material budget of such a device.
  This document is organised as followed:
  the theoretical context is presented in chapter~\ref{chap:SM}, with an overview of the \gls{SM} and theories beyond the \gls{SM}.
  %Chapter~\ref{chap:SM} presents the theoretical context, with an overview of the \gls{SM} and theories beyond the \gls{SM}.
  Chapter~\ref{chap:ILC} presents the future linear collider, the \gls{ILC} focusing especially on one of the experiments, the \gls{ILD}.
  Chapter~\ref{chap:phyics} introduces the different physics studies that will be performed at the \gls{ILC} and focus especially on a possible analysis of the $\nu\overline{\nu}H$ channel at the \gls{ILC}.
  In chapter~\ref{chap:vxd}, the different \gls{VXD} for the \gls{ILD} are presented, as well as a description of the \gls{PLUME} collaboration and the status of the detectors produced.
  The three last chapters are devoted to the studies performed during this thesis.
  In chapter~\ref{chap:labTests}, the validation in the laboratory of the different \gls{PLUME} modules are reported.
  Chapter~\ref{chap:deformation} presents the observation of the ladder deformation during a test beam campaign which was done in 2011 at \gls{CERN}. 
  It also shows the benefits of a double-sided measurement compared to a single-sided ladder.
  Chapter~\ref{chap:X0} deals with the measurement of the radiation length of the first fully working \gls{PLUME} ladder, which has a weighted material budget ($\rm{X_0}$) estimated to be $0.65~\%~\rm{X_0}$.
  Finally, the conclusion summarises the work performed during the thesis and the outlook is discussed.
