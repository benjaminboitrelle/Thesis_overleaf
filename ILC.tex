\chapter{The future of high-energy physics: the ILC}

 %Introduction on the limits of the LHC and why a new colliding experiment is needed
 % First part: describing the main characteristics of the ILC (energy scale, legnth, luminosity)
 %   -> Design (e+/e- sources, damping ring, main linacs, beam delivery system and interaction region )
 %   -> Beam properties
 %   -> Background: beam-beam interaction (luminosity enhancement ~2 BUT hard bremstrahlung that degrades the energy spectrum), pair background (coherent and incoherent production of e+e-)
 % Second part: Detector
 %   -> Option for two experiments (PUSH-PULL)
 %   -> Main differences between the ILD and the SiD
 %   -> Talking about the ILD (VTX, SIT, TPC, f)


  The LHC has started in 2008. The energy scales reaches by the LHC has permitted the discovery of the Higgs boson, but also a spin-3 particles with the LHCb experiment.
  The LHC was designed to be a really challenging machine with a record centre-of-mass energy of collisions on Earth.
  It consists of ??? kilometer ring that provides collision of 14 TeV. 
  Unfortunately, the collision are not well known due to the structure of the proton. 
  Only an estimation of the energy collision is know and the huge QCD background limit the precision of measurement.
  The LHC is the most powerful tool for direct discovery

  A lepton collider is running at lower energy to perform high sensitivity investigations.
  
  The high energy physics community is thinking about a future accelerator which would work as a complement of the LHC.


  Since the beginning of the 2000, the high energy physics community is thinking about a future accelerator to work in complement of the LHC.


  Two projects are considered for the future: the ILC and CLIC (far future).
  Due to the purpose of this thesis, only the ILC project will be presented. 
  
  
   
  \section{The ILC machine}

  \begin{figure}
    \centering
    \missingfigure{Design of the ILC}
    \caption{Basic design of the International Linear Collider}
    \label{fig:ILC}
  \end{figure}
  \section{The SiD}

  \section{The ILD}
