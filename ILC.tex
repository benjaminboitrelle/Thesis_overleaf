\chapter{The future of high-energy physics: the ILC}

 %Introduction on the limits of the LHC and why a new colliding experiment is needed
 % First part: describing the main characteristics of the ILC (energy scale, legnth, luminosity)
 %   -> Design (e+/e- sources, damping ring, main linacs, beam delivery system and interaction region )
 %   -> Beam properties
 %   -> Background: beam-beam interaction (luminosity enhancement ~2 BUT hard bremstrahlung that degrades the energy spectrum), pair background (coherent and incoherent production of e+e-)
 % Second part: Detector
 %   -> Option for two experiments (PUSH-PULL)
 %   -> Main differences between the ILD and the SiD
 %   -> Talking about the ILD (VTX, SIT, TPC, f)

  Since the beginning of the 2000, the high energy physics community is thinking about a future accelerator to work in complement of the LHC.


  Two projects are considered for the future: the ILC and CLIC (far future).
  Due to the purpose of this thesis, only the ILC project will be presented. 
  
  
   
  \section{The ILC machine}

  \begin{figure}
    \centering
    \missingfigure{Design of the ILC}
    \caption{Basic design of the International Linear Collider}
    \label{fig:ILC}
  \end{figure}
  \section{The SiD}

  \section{The ILD}
