\chapter[The ILC]{The future of high-energy physics: the International Linear Collider}
\label{chap:ILC}


 %Introduction on the limits of the LHC and why a new colliding experiment is needed
 % First part: describing the main characteristics of the ILC (energy scale, legnth, luminosity)
 %   -> Design (e+/e- sources, damping ring, main linacs, beam delivery system and interaction region )
 %   -> Beam properties
 %   -> Background: beam-beam interaction (luminosity enhancement ~2 BUT hard bremstrahlung that degrades the energy spectrum), pair background (coherent and incoherent production of e+e-)
 % Second part: Detector
 %   -> Option for two experiments (PUSH-PULL)
 %   -> Main differences between the ILD and the SiD
 %   -> Talking about the ILD (VTX, SIT, TPC, f)

  Since 2008, the Large Hadron Collider (LHC) is actually the most powerful tool in high energy physics to have a better understanding of the universe, particularly with the discovery in 2012 of a new particle compatible to the boson predicted by the spontaneous symmetry breaking of the SM \cite{Aad2012, Chatrchyan2012}.
  Although the LHC is an impressive machine able to reach the highest energy scales of collision available on Earth with a centre-of-mass energy of 14 TeV, the complex environment of the events generated hides the access to some fundamental parameters. 
  To achieve more precise measurements of the Higgs boson, but also to test the validity of the SM and other physics theories introduce in the chapter \ref{chap:SM}, the high energy physics community has merged on the necessity to build a linear electron-positron collider, that will work as a complementary machine to the LHC.
  
  This chapter will explain the motivations to invest a huge amount of money in a new great world project. It will present the complementary nature of the lepton and hadron colliders and the main advantages of the lepton collisions will be discussed.
  After giving an overview of the ILC with its basic design and the experiment models, we will focus on the design of one of the experiments: the International Large Detector (ILD).

 \minitoc
  
 % It consists of ??? kilometer ring that provides collision of 14 TeV. 
 % Unfortunately, the collision are not well known due to the structure of the proton. 
 % Only an estimation of the energy collision is know and the huge QCD background limit the precision of measurement.
 % The LHC is the most powerful tool for direct discovery

 % A lepton collider is running at lower energy to perform high sensitivity investigations.
 % 
 % The high energy physics community is thinking about a future accelerator which would work as a complement of the LHC.


 % Since the beginning of the 2000, the high energy physics community is thinking about a future accelerator to work in complement of the LHC.


 % Two projects are considered for the future: the ILC and CLIC (far future).
 % Due to the purpose of this thesis, only the ILC project will be presented. 
  
  
   
  \section{The ILC machine}
 
  Before to introduce the ILC machine and explain the reasons to invest in such a project, I will give an overview of the LHC abilities and the limits of this giant tool.
  The most impressive accelerator ever built is located at CERN in Switzerland. 
  The 26.659 kilometers circumference ring, crossing the french border, is providing since 2015 a centre-of-mass energy of 14 TeV.
  They are four experiments currently running: LHC-b, ALICE, CMS and ATLAS.
   
    \subsection{Advantages of lepton collider}

    Collisions with leptons are well defined compared to quarks collisions. 
    Indeed, leptons are structureless objects that leads to know exactly the initial four-vector momentum to reconstruct fully the event, whereas the quark are made of partons.
    Due to their composite structure, only a fraction of the proton is part of the collision. 
    The partons that do not participate to the collisions contribute to a parasitic soft interaction, the QCD background.
    To get a cleaner signal at the ILC, a trigger is implemented.
    As the ILC has a direct access to a fully reconstructed event, no trigger is needed.

    It is just fucking bullshit what I am writing right now. 

    
    % - Leptons are structureless -> Well known 4-vector momentum to reconstruct fully the event
    % - Quarks are made of partons => Impossible to know exactly the 4-vector momentum
    % - Partons that do not participate to the collisions contribute to parisitic soft interaction (QCD bg)
    %       => Hide elementary processes (small polar angle)
    %       => Needs implementation of trigger
    % - Beam energy tunable and polarisation of e+/e- can enhance the signal and suppress the background cross-section 
    \subsection{}
  

  \begin{figure}
    \centering
    \missingfigure{Design of the ILC}
    \caption{Basic design of the International Linear Collider}
    \label{fig:ILC}
  \end{figure}
  \section{Design of the ILC}
    \subsection{Baseline design}
    \subsection{Beam parameters}
    \subsection{Background}

  \section{The ILC detectors concept}

    \subsection{Overview of the two experiments}
    
  Although the ILC is a linear collider with only one beam line, two experiments will participate.
  The duplication of the beam to get two permanent fixed detectors is too expensive and difficult to set-up.
  Instead, the collaboration has the idea to have only one interaction point.
  A push/pull system will allow to change the detector 

  Two detector designed with complementary features have been developed. 
  Their designs will reach the ILC precision measurements and search for new physics. 
  The two experiments are SiD and ILD


      \subsection{SiD VS ILD}

    \subsection{I don't know th title name}
      \subsubsection{Vertex detector}
      \subsubsection{Tracking}
      \subsubsection{Calorimeters}
