\chapter[The ILC]{The future of high-energy physics: the International Linear Collider}
\label{chap:ILC}


 %Introduction on the limits of the LHC and why a new colliding experiment is needed
 % First part: describing the main characteristics of the ILC (energy scale, legnth, luminosity)
 %   -> Design (e+/e- sources, damping ring, main linacs, beam delivery system and interaction region )
 %   -> Beam properties
 %   -> Background: beam-beam interaction (luminosity enhancement ~2 BUT hard bremstrahlung that degrades the energy spectrum), pair background (coherent and incoherent production of e+e-)
 % Second part: Detector
 %   -> Option for two experiments (PUSH-PULL)
 %   -> Main differences between the ILD and the SiD
 %   -> Talking about the ILD (VTX, SIT, TPC, f)

  Since 2008, the Large Hadron Collider (LHC) is actually the most powerful tool in high energy physics to have a better understanding of the universe, particularly with the discovery in 2012 of a new particle compatible to the boson predicted by the spontaneous symmetry breaking of the SM \cite{Aad2012, Chatrchyan2012}.
  Although the LHC is an impressive machine able to reach the highest energy scales of collisions available on Earth with a centre-of-mass energy of 14 TeV, the complex environment of the events generated hides the access to some fundamental parameters. 
  To achieve more precise measurements of the Higgs boson, but also to test the validity of the SM and other physics theories introduce in the chapter \ref{chap:SM}, the high energy physics community has merged on the necessity to build a linear electron-positron collider, that will work as a complementary accelerator to the LHC.
  
  This chapter will explain the motivations to invest a huge amount of money in a new great world project. It will present the complementary nature of the lepton and hadron colliders and the main advantages of the lepton collisions will be discussed.
  After giving an overview of the ILC with its basic design and the experiment models, we will focus on the design of one of the experiments: the International Large Detector (ILD).

 \minitoc
  
  \section{The ILC machine}
 
  Before to introduce the ILC machine and explain the reasons to invest in such a project, I will give an overview of the LHC abilities and the limits of this giant machine.
  The most impressive accelerator ever built is located at CERN in Geneva, Switzerland. 
  It is the world's largest particle accelerator, with a circumference ring of nearly 27 kilometers, straddling the Swiss and French borders.
  It is designed to collide two counter rotating beams of protons or heavy ions, with a possibility to reach centre-of-mass energies of 14 TeV with a high peak luminosity of $10^{34} \text{ cm}^2 \text{s}^{-1}$.
  The goals of the LHC are to perform further tests on the SM, search for new forces or produce dark matter candidates. 
  Indeed, the collider covers a wide energy range at the constituent level while running at a fixed beam energy.
  Unfortunately, the measurements can not reach the highest precision.
  
  Complementary to a discovery machine such as the LHC, a machine to perform precise measurement should be built: the lepton collider.

    \subsection{Advantages of a linear lepton collider}
    
    First of all, during each collision at an hadron collider, only a part of the total centre-of-mass energy is available for the process evolved, so the initial four-vector momentum is not known. 
    By colliding leptons, which are structureless objects, the full centre-of-mass energy is available for the elementary process. 
    The initial four-vector momentum of an interaction is exactly known, hence the event can be fully reconstructed.

    Secondly, with a lepton collider, the beam energy is tunable and both electron and positron beams can be polarised. 
    The selection of an appropriate polarisation can boost the signal and suppress the background cross-section. 

    Thirdly, as seen on the first point, at the LHC, only a fraction of the partons are contributing to the interesting process. 
    The proton-proton interaction cross section is dominated by inelastic background QCD processes.
    The signal event is then accompanied by large backgrounds produced by the interaction of other partons collisions.
    This background has an impact on the detector design (high radiation tolerance and selective trigger implementation) and masks the elementary process of interests. 
    The lepton colliders do not suffer from this kind of background and at similar energies, the event rate is lower as those of hadron colliders.
    So, the detectors do not have to handle extreme data rates and they can be use without any trigger.
    This will allow to get a better sensitivity to any possible signature of new physics.

    % - Leptons are structureless -> Well known 4-vector momentum to reconstruct fully the event
    % - Quarks are made of partons => Impossible to know exactly the 4-vector momentum
    % - Partons that do not participate to the collisions contribute to parisitic soft interaction (QCD bg)
    %       => Hide elementary processes (small polar angle)
    %       => Needs implementation of trigger
    % - Beam energy tunable and polarisation of e+/e- can enhance the signal and suppress the background cross-section 
    
    % LINEAR VS CIRCULAR COLLIDER!
    \subsection{ }
  
  blablabla

  \begin{figure}
    \centering
    \missingfigure{Design of the ILC}
    \caption{Basic design of the International Linear Collider}
    \label{fig:ILC}
  \end{figure}

  \section{Design of the ILC}
    \subsection{Baseline design}
    \subsection{Beam parameters}
    \subsection{Background}

  \section{The ILC detectors concept}

    \subsection{Overview of the two experiments}
    
  Although the ILC is a linear collider with only one beam line, two experiments will participate.
  The duplication of the beam to get two permanent fixed detectors is too expensive and difficult to built.
  Instead, the collaboration has the idea to have only one interaction point.
  A push/pull system will allow to change the detector 

  Two detector designed with complementary features have been developed. 
  Their designs will reach the ILC precision measurements and search for new physics. 
  They are optimised for the particle flow algorithm (PFA) to measure the final states with high accuracy.
  This leads to a high hermeticity, high granular calorimeters and excellent tracking and vertexing. 
  The two experiments are the Silicon Detector (SiD) and the International Large Detector (ILD) and a global comparison of the two detectors is going to be presented on the next subsection.

      \subsection{SiD VS ILD}

    \subsection{I don't know th title name}
      \subsubsection{Vertex detector}
      \subsubsection{Tracking}
      \subsubsection{Calorimeters}
