\chapter{Secrets of the nature}

 \minitoc
  \section{The Standard Model}

    \subsection{Introduction}
    
    %\lettrine{\textgoth{T}}{he} 
    The Standard Model (SM) is a theory that describes the fundamental structure of the matter surrounding us. 
    It is one of the most successful achievement in modern physics.
    The elegant theoretical framework of the SM is able to provide good explanation of experimental results, but is also able to predict a wide variety of phenomena.

    The SM depicts the interactions between the fundamental constituents of matter, called particles.
    From a quantum point of view, a particle is defined by its intrinsic angular momentum, called spin. 
    That quantum number is a key to distinguish between the particle of 'matter' and the 'carrier force' particle.
    
    The half integers spin particles are obeying to the Fermi-Dirac statistics and are submitted to the Pauli exclusion principle:
    They can not occupied the same quantum state at the same time.
    These particles are called fermions.
    They are the constituents of the matter and are to the number of twelve.

    The fermions are divided into two categories: the leptons and the quarks. 
    The leptons are to the number of six: three charged particles and three neutral particles called neutrino.
    The first fundamental particle discovered in particle physics was the electron (e$^-$) at the end of the $19^{th}$ century.
    The two other charged leptons were discovered in 1937 for the muon ($\mu$) and in 1975 for the tau ($\tau$):
    Three neutrinos are associated to the three flavored leptons: the electron neutrino ($\nu_e$) discovered in 1953, the muon neutrino ($\nu_{\mu}$) in 1962 and the tau neutrino ($\nu_{\tau}$) discovered in 2000.

    The quarks are to the number of six.
    They can't be find in alone in the nature.
    They are carrying a quantum number: the color.
    The color quantum numbers are green, blue and red (and the anti-color associated).
    They are always in a bounded state to form composite particles that are colorless and are called hadrons.
    A quark and an anti-quark form an integer spin composite particle, called meson.
    Three quarks bounded together are called baryons. The most known baryons are the proton and the neutron.
    They are made of the up quark (u) and the down quark (d).
    The other quarks were discovered in the second half of the 20$^{th}$ century.
    The strange quark (s) was discovered in 1968, followed by the charm quark (c) in 1974.
    Then, the bottom quark or beauty quark (b) was discovered in 1977.
    The last quark discovered wad the top quark (t) in 1995.  

    The fermions are also divided into three categories that depends on the mass of the particle.
    The categories are called fermions.
    The first generation of particles is composed of the electron, the electron neutrino, the u and d quarks. 
    They form the ordinary matter.
    The two other generations are particle that can be found in cosmic rays or in collision with accelerators.
    All the fermions and their properties is summarised in the table \ref{tab:fermions}.

    \begin{table}[!h]
      \begin{center}
        \begin{tabular}{c c c c c c c}
        \hline %----------------------------
        Type & Family & Particle  & L & B & Q$_e$ & Mass  \tabularnewline
        \hline %----------------------------
        \hline %----------------------------
        \multirow{6}*{Leptons} & \multirow{2}*{1$^{st}$}    & $e$       & 1 & 0 & -1    & 511 keV \tabularnewline
                               & & $\nu_e$   & 1 & 0 & 0     & < 2 ev \tabularnewline
                               & \multirow{2}*{2$^{nd}$}    & $\mu$     & 1 & 0 & -1    & 105.66 MeV \tabularnewline
                               & & $\nu_{\mu}$ & 1 & 0 & 0   & < 2 eV \tabularnewline
                               & \multirow{2}*{3$^{rd}$}    & $\tau$   & 1 & 0 & -1     & 1.78 GeV \tabularnewline
                               & & $\nu_{\tau}$ & 1 & 0 & 0  & < 2 eV \tabularnewline
        \hline %----------------------------
        \hline %----------------------------
        \multirow{6}*{Quarks} & \multirow{2}*{1$^{st}$} & u & 0 & 1 & 2/3 & $2.3^{+0.7}_{-0.5} MeV$\tabularnewline
                              & & d & 0 & 1 & -1/3 & $4.8^{+0.5}_{-0.3} MeV$\tabularnewline
                              & \multirow{2}*{2$^{nd}$} & s & 0 & 1 & -1/3 & $ 95\pm 5 MeV $ \tabularnewline
    		                  & & c & 0 & 1 &  2/3 & $1.275 \pm 0.0025 GeV $ \tabularnewline
                              &\multirow{2}*{3$^{rd}$} & b & 0 & 1 & -1/3 & $4.66 \pm 0.03 GeV$ \tabularnewline
        					  & & t & 0 & 1 & 2/3 & $ 173.21 \pm 0.51 \pm 0.71 GeV$\tabularnewline
        \hline %----------------------------        
        \end{tabular}
      \end{center}
        \label{tab:fermions}
            \caption{Summary of the 12 fermions.}
    \end{table}

    The second kind of particles are integer spins particles and are labelled bosons or gauge bosons.
    They are following the Bose-Einstein statistics 
    It means that the bosons are not limited to single occupancy of the same state as the fermions.
    The bosons are the mediator of the four fundamental interactions.    
    
    The electromagnetic interaction (EM) is mediated by the photon $\gamma$, a massless and chargeless particle of spin 1.
    The EM is responsible for the interaction between two charged particles.
    The weak interaction which is responsible of the $\beta$ radioactive decay (a nucleon is able to transform into an other one with the emission of a lepton and a neutrino).
    The gauge bosons bosons associated to the weak interaction are the neutral electrical charged boson $Z^0$, and two electrical charged one $W^+$ and $W^-$.
    The strong interaction is mediated by eight gauge bosons, the gluons.
    It is responsible for the nucleus cohesion and the hadrons.
    The last force is the gravitational interaction but it is not included into the SM.
    Trying to find a framework where the equation of the general relativity used to describe the macro world and the equation of the quantum mechanics describing the micro world is a difficult challenge.
    From a quantum theory, the boson associated to the gravitational force might be the graviton, a spin 2 particle. 

    The Higgs boson (H) is a particle predicted by the S.M and is the only one to have been found in 2012 at the LHC. 
    It is the gauge boson of the Higgs mechanism.
    This mechanism is the mass generator of particles and will be presented later REF TO LATER
    %assumed to be responsible for the generation of the masses and can be explained by the electroweak symmetry breaking.
    
    The table \ref{tab:boson} summarises the different bosons of the SM.
  \begin{table}[!h]
    \begin{center}
        \begin{tabular}{c c c c c}
        \hline %----------------------------
        Force & Gauge bosons & Mass (GeV/$c^2$) & Electric charge & Range \tabularnewline
        \hline %----------------------------
        \hline %----------------------------
        Electromagnetic & $\gamma$ & 0 & 0 &\tabularnewline  
        \multirow{2}*{Weak} & $Z^0$ & 91.1876 $\pm$ 0.0021& 0 &\tabularnewline
             & $W^{\pm}$ & 80.3980 $\pm$ 0.0250 & $\pm 1$  &\tabularnewline 
        Strong & g (8 gluons) & 0 & 0 \tabularnewline
        \hline %----------------------------
        \hline %----------------------------
            & H & 125 GeV & 0 & \tabularnewline
        \hline %----------------------------
        \end{tabular}
    \end{center}
    \label{tab:boson}
    \caption{Summary of the bosons}
  \end{table}
  \todo{Add range of particles}

    
    \begin{figure}[h]
    \centering
    \missingfigure{Particles and boson}
    \caption{}
    \label{fig:partInterac}
    \end{figure}
      
    \subsection{Quantum Field Theory}

	The mathematical basis of the SM is the Quantum Field Theory (QFT). All the interactions are described by the gauge group 
    
      \begin{equation}
    	SU_C(3) \otimes SU_L(2) \otimes U_Y(1)
	  \end{equation}
    
    The gauge theory is invariant under a continuous set of local transformation.
    Taking the gauge symmetries and the least action into account, physicists were able to set up equations that describe the dynamic of the interactions by Lagrangian.
    The steps to build Lagrangian for the three forces and the unification of the EM and weak interactions are going to be presented. 
    
      \subsubsection{Quantum Electrodynamic}

      The Quantum Electrodynamic (QED) is the QFT used to described the electromagnetic interactions using a $U(1)$ gauge group.
     The Quantum Electrodynamic (QED) is the QFT that combines the electromagnetism formalism and the quantum mechanics formalism to describe the interaction thanks to a relativistic Lagrangian.
     As the charge $Q_e$ of the electron is invariant on every part of the Universe, the QED Lagrangian should be invariant under some transformation.
     The $U(1)$ gauge group is a unitary group of one dimension that is invariant under space transformation.

      Lets first consider the Dirac equation for a free fermion:
      
      \begin{equation}
        \mathcal{L}_{Dirac} = \overline{\Psi}\left(x\right) \left(i \gamma^{\mu}\partial_{\mu} - m \right) \Psi\left(x\right)
      \end{equation}

      The Lagrangian is invariant under global U(1) transformation:

      \begin{equation}
            \begin{array}{rrccr}
             \Psi \left(x \right) & \rightarrow & \Psi^{'} \left(x \right)  & = & e^{-i\alpha} \Psi\left(x\right) \\
             \overline{\Psi}\left(x\right) & \rightarrow & \overline{\Psi}^{'}\left(x\right) & = & e^{i\alpha}  \overline{\Psi}\left(x\right) \\
            \end{array}
      \end{equation}

      The corresponding local symmetry is:

      \begin{equation}
            \begin{array}{rcccr}
             \Psi\left(x\right) & \rightarrow & \Psi^{'} \left(x \right) & = & e^{-i\alpha(x)} \Psi\left(x\right) \\
             \overline{\Psi}\left(x\right) & \rightarrow & \overline{\Psi}^{'}\left(x\right) & = & e^{i\alpha(x)}  \overline{\Psi}\left(x\right) \\
            \end{array}
      \end{equation}

      Considering the local symmetry, the mass term of the Lagrangian REF-Dirac-Lagrangian stays invariant but the term which contains the derivative is not anymore:
      By introducing a material derivative that includes a gauge field $A_{\mu}$, it is possible to keep the derivative invariant under local gauge transformation:

      \begin{equation}
        D_{\mu} \Psi\left(x\right) =  \left(\partial_{\mu} - i Q_e A_{\mu}\right) \Psi\left(x\right)
      \end{equation}

     The gauge field is not yet a dynamic field. To get a physical gauge field, a kinetic term should be added to the equation.
     This gauge invariant term that includes derivative from the $A_{\mu}$ field is:
    
     \begin{equation}
       F_{\mu \nu} \ = \ \partial_\mu A_\nu - \partial_\nu A_\mu
     \end{equation}

     The Lagrangian that is local invariant, is the one that describes the QED:

    \begin{equation}
    	\mathcal{L}_{QED} =  \overline{\Psi}\left(x\right)\left( i \gamma^\mu D_\mu - m \right) \Psi\left(x\right) - \frac{1}{4}F_{\mu \nu}\left(x\right) F^{\mu \nu}\left(x\right)
    \end{equation}

    A mass term $m A_{\mu} A^{\mu}$ for the field $A_{\mu}$ is missing because it is not gauge invariant. It can be explain by the fact that the photon is massless.

    \todo{Add details on the QED: coupling...}

    \subsubsection{Weak interaction}

    The $\beta$ decay of the nuclei is explained by the weak interaction, a theory initiated by Fermi.
    The weak interaction is different from the EM and strong interaction.
    The fact that particles messengers are massive bosons seems to be inconsistent with the gauge invariance principle.
    Also, the coupling of quarks and leptons are different.
    The QED and QCD are parity conserving theories of charged particles, whereas the weak interaction violate parity.
    The experiment lead by Madame Wu is a proof of the parity violation in weak interaction. 
    Only the left-handed fermions chiralities participate in the weak interaction.
    
    The weak interaction explains the $\beta$ decays of the nuclei.
    It couples the left-handed particles and the right-handed anti-particles.
    
    The weak interaction is described by a non-Abelian group\footnote{A group is non-Abelian when the elements of the group do not commutate.}, the SU(2) symmetry group.
    
    \subsubsection{Quantum Chromodynamics}
    
    The Quantum Chromodynamics (QCD) is the quantum field theory of the strong interaction.
    In this model, the interaction is due to a SU(3) gauge group. 
    It produces 8 gauge fields called gluons.
    The spinors of this theory are the six quarks that form a triplet with respect to the gauge symmetry.

    The SU(3) gauge group is a group of $9 - 1 = 8$ real parameters and of 8 generators. 
    Those generators are the Gell-Mann matrices. 
    The normalised generators are defined by: 
    
    \begin{equation}
        T^a = \frac{1}{2}\lambda^a
    \end{equation}

    The structure constant $f^{abc}$ can be expressed as:

    \begin{equation}
        if^{abc} = 2 Tr([T^a,T^b]T^c)
    \end{equation}
     
    Some theories arguments and the results of experiments in high energy physics ask to introduce six spinor fields, the quarks.
    Each of them are considered as a triplet state with respect to the SU(3) group:

    \begin{equation}
      q_i = 
        \begin{pmatrix}
          q_i^1 \\
          q_i^2 \\
          q_i^3 \\
        \end{pmatrix}
     \end{equation}
    
    where $q_i$ are the six quarks.
    These quarks can appeared in three different states, called color and that are named red, blue and green.

    The local gauge symmetry U(1) should be included into the SU(3) group.
    
    The gauge field $A_{\mu}$ can be introduced in the group:
    
    \begin{equation}
      A_{\mu} = g_S A^a_{\mu}\frac{\lambda^a}{2}
    \end{equation}
     
    with a = 1,...,8 and corresponds to the 8 gluons.
    A mass term  $m_g A^{\mu}_a A^a_{\mu}$ would not be gauge invariant, that implies the gluons are massless.

    The material derivative is then:

    \begin{equation}
      D_{\mu} = \partial_{\mu} - i A_{\mu} = \partial_{\mu} - i g_S A^a_{\mu} \frac{\lambda^a}{2}
    \end{equation}

    The QED field $F_{\mu \nu}$ is not gauge invariant in QCD.
    Nevertheless an additional term to obtain gauge invariant field tensor can be introduced:
    
    \begin{equation}
      G^a_{\mu \nu} = \left( \partial_{\mu} A^a_{\nu} - \partial_{\nu} A^a_{\mu} \right) + g_S f^{abc} A^b_{\mu} A^c_{\nu}
    \end{equation} 

    Finally, the QCD Lagrangian is given by:

    \begin{equation}
      \mathcal{L} = \sum_{i=1}^6  \bar{q_i} \left(i \gamma^{\mu}D_{\mu} -m_i \right)q_i - \frac{1}{4} G_{\mu \nu}^{a} G_{a}^{\mu \nu}
    \end{equation}
    
    \subsubsection{Glashow-Weinberg-Salam model}
  
    Lagrangian EW goes here: 

  \section{The Higgs physics}

    The Standard Model constitutes one of the most successful achievement in modern physics.
    One of its strength is to provides a elegant theoretical framework to describe the known experimental facts about particles, but also it was able to predict 
    the existence of a mechanism to generate the particle masses via the Higgs mechanism.

    With the lagrangians described above, the gauge boson were considered as massless fields.
    The electroweak interaction does not allow a $m\overline{\Psi}\Psi$ term because it does not transform as a scalar under $SU(2_L)\otimes U(1)g$.
    The $m^2A_{\mu} A^{\mu}$ violates the gauge invariance of the Lagrangian.
    That section will present the mechanism to generate fermions and boson masses.

    \subsection{Symmetry Breaking mechanism}
    
    As we have seen with the Lagrangian of the QED and QCD, the bosons generated are massless. Nevertheless, the W boson and the Z bosons have a mass. 
    The origin of the fermions masses is solved in the SM thanks to the Higgs-Englert-Brout mechanism.

    Lets consider first a complex scalar doubled field $\phi$ and a lagrangian, invariant under local gauge transformations:

    \begin{equation}
      \mathcal{L} = \left(D^{\mu} \phi \right)^{\dagger} \left( D_{\mu} \phi \right) - V(\phi)
      \label{eq:lagrangianHiggs}
    \end{equation}             

    The first term corresponds to the kinetic term of a scalar field.
    The second term is a scalar potential that is invariant under $SU(2)_L$ and it is:

    \begin{equation}
      \begin{array}{lr}
        V(\phi) = \mu^{2}\phi^{\dagger}\phi + \lambda \left(\phi^{\dagger}\phi\right)^2  & , with \ \lambda > 0 \\
      \end{array}   
    \end{equation}
    
    Depending if $\mu^{2}$ is positive or negative, the potential can take two forms.
    If $\mu^{2} > 0$, the potential has a minimum at $\phi = 0$ and describes a scalar particle with a mass $\mu$ and a quartic self coupling.
    As the transformation $\phi \rightarrow  - \phi$ is respected, this solution is a symmetric one.

    When $\mu^{2} < 0$, there is not a unique ground state for this system but multiple states with the same vacuum energy.
    
    the minima is obtained for those fields configurations:

    \begin{equation}
      v = \sqrt{\frac{- \mu^2}{2\lambda}} > 0
    \end{equation}

    This minima will be considered as a vacuum state.
    Lets write a new field:

    \begin{equation}
      \phi(x) = v + h(x)
      \label{eq:fieldHiggs}
    \end{equation}

    The value $v$ is given by one of the solution from equation REFXXX, and $h(x)$ is new physical variable.
    If the equation \ref{eq:fieldHiggs} is introduced into the equation \ref{eq:lagrangianHiggs}, the lagrangian takes the form below:

    \begin{figure}[h]
    \centering
    \missingfigure{Mexican hat}
    \caption{Scalar potential}
    \label{fig:scalarPotential}
    \end{figure}

    \subsection{Higgs mechanism}

    Here talk about the Higgs mechanism but may be added in a Higgs chapter...


   \subsubsection{Higgs Decays}

   As explained before, the Higgs boson couples to all particles of the Standard Model.
   The figure \ref{fig:higgsBR}, the branching ratios for different decay modes are shown as a function of the mass of the Higgs boson.
   For a Higgs mass of 125.5 GeV, a large number of decays are accessible to experiments

    \begin{figure}[h]
    \centering
    \missingfigure{Higgs Branching ratio}
    \caption{The branching ratio of the Higgs}
    \label{fig:higgsBR}
    \end{figure}

   \subsubsection{Higgs production at the ILC}
    
    \begin{figure}[h]
    \centering
    \missingfigure{Higgs Production}
    \caption{Higgs production}
    \label{fig:higgsBR}
    \end{figure}
      
  \section{Limitations of the Standard Model}

  The SM is one of the best theory to describe the infinity and beyond. 
  Nevertheless, it can not explain all the mysteries in the universe. 
  This theory conforms experimental data in high energy physics but is unsatisfactory.
  
  The SM does not explain the quantum numbers. There was 19 free parameters that were imposed.0 

  Physicists have to test the limits of the Model and find a theory that is working beyond the SM. 

  Although the SM is able to describe some phenomenon up to about a hundred GeV, it is not the most perfect model.

    \subsection{Gravitation}
    There is no viable theory that can include the SM and the general relativity.

    \subsection{Neutrino mass}

    The neutrinos defined buy the SM are assumed to be exactly massless.
    Nevertheless at the end of the 90's, the Super Kamiokande experiment had surprising results.
    There was a lack on the expected solar and atmosphere neutrino flux. 
    The results was interpreted by an oscillation of neutrinos between the three leptonic flavors.
    However, the oscillation is possible only if the neutrino has a mass.
    The neutrino oscillation can be considered as a proof of physics beyond the SM.

    \subsection{Free parameters and generation}

    The SM does not predict the three generations and the structure (2 leptons and 2 quarks per generation). 
    Also, the SM lacks to explain the number of parameters (the 19 quantum numbers).

    \subsection{Hierarchy problem}

    \subsection{Matter-antimatter asymmetry}

    As discussed before, the SM defines equal number of particles and anti-particles. 
    However, the entire universe is only made of matter. 
    Why preference to matter and no one to anti-matter? 

    \subsection{Dark Matter}
    
    Nowadays only twelve particles (plus the anti-particles associated) have been observed. 
    The idea of dark matter comes from the way we estimate the mass of galaxy.

