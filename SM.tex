\chapter{From the discovery of the electron to the high-energy physics}

  \section{The Standard Model}

    \subsection{Introduction}
    
    The Standard Model (SM) is a theory that describes the fundamental structure of matter. 
    All of the matter found in the universe is made up of fermions, fundamental particles of spin 1/2 and that are to the number of 12.
    It can be found also some intermediate gauge bosons, that are mediators of interactions and are particles of spin 1. 

    Higgs boson is a particle predicted particle by the S.M and is the only one to have been found in 2012 at the LHC. It is a product of the Higgs mechanism which is assumed to be responsible for the generation of the masses and can be explained by the electroweak symmetry breaking.
    
    The fermions are elementary particles of spin 1/2. They are to the number of 12 and they have 12 anti-particles associated.  This anti-fermions have the same mass but an opposite electrical charge and a magnetic dipole moment. The fermions are divided in two categories: the leptons and the quarks.
    
    The leptons are free particles. They can be found alone in the nature. They can feel the electromagnetic and the weak interactions. The neutrinos are also almost mass-less leptons but can interact only via weak interaction.
    
    The quarks are particular fermions. They were never observed alone in the nature. They are always in bound states, called hadrons. The quarks are carrying quantum color numbers. This quantum color numbers are green, blue, red plus the anti-color associated $\overline{green}$ $\overline{blue}$, $\overline{red}$. Only colorless particle can be observed. They are in double quarks state, called mesons or triple quarks state, called baryons. The hadrons are sensible to the electro-weak interaction but also to the strong interaction.
    
    The S.M. tempts to describe also the fundamental interaction of the particles. It has been able to describe only three out of four interaction: the electromagnetic, the weak and the strong interactions. Some theories try to explain the gravity but this theme will not be discussed.

	The interaction between fermions is carried by particle of spin 1 called gauge boson. The electromagnetic interaction is mediated by the photon $\gamma$, a mass-less boson.
    The weak interaction is mediated by three massive bosons: an neutral electrical charged boson $Z^0$, and two electrical charged $W^+$ and $W^-$.
    The strong interaction is mediated by eight gauge bosons, the gluons. 

    In this theory, the matter is made up of fermions (spin 1/2) which interact via bosons. 
    Bosons are gauge field, they propagate the interaction and generate particles mass.
    They are two types of fermions: the leptons and the quarks. 
    The leptons are made of six particles plus the  six anti-particles associated. 
    Neutrons are associated to the leptons.
    A quantum number describes those particles: L
    Leptons interact only via weak interaction.

    The quarks are six plus six anti-quark. 
    They are the components of neutron and protons.
    Lonely quarks were never observed in the nature.
    The quarks have a charge color : green, red, blue.
    The quarks are confined into "white color" objects call hadrons.
    The hadrons are subdivided into three categories: the mesons, the baryons and the anti-baryons.
    The mesons are made of a quark and a anti-quark are always of integer spin.
    The baryons (or anti-baryons) are made of three quarks, like the proton or neutron.
    A quantum number is associated to the quarks: the baryon number B which is conserved for every interaction and has the same properties as the L number.


    \begin{center}
        \begin{tabular}{c|c|c|c|c|c}
        \hline %----------------------------
        Family & Particle  & L & B & Q$_e$ & Mass  \tabularnewline
        \hline %----------------------------
        \hline %----------------------------
        \multirow{2}*{1$^{st}$}    & $e$       & 1 & 0 & -1    & 511 keV \tabularnewline
               & $\nu_e$   & 1 & 0 & 0     & < 2 ev \tabularnewline
        \multirow{2}*{2$^{nd}$}    & $\mu$     & 1 & 0 & -1    & 105.66 MeV \tabularnewline
               & $\nu_{\mu}$ & 1 & 0 & 0   & < 2 eV \tabularnewline
        \multirow{2}*{3$^{rd}$}    & $\tau$   & 1 & 0 & -1     & 1.78 GeV \tabularnewline
               & $\nu_{\tau}$ & 1 & 0 & 0  & < 2 eV \tabularnewline
        \hline %----------------------------
        \end{tabular}
    \end{center}
    
    \begin{center}
        \begin{tabular}{c|c|c|c|c|c}
        \hline %----------------------------
        Family & Particle  & L & B & Q$_e$ & Mass  \tabularnewline
        \hline %----------------------------
        \multirow{2}*{1$^{st}$} & u & 0 & 1 & 2/3 & \tabularnewline
                               & d & 0 & 1 & -1/3 & \tabularnewline
        \multirow{2}*{2$^{nd}$} & s & 0 & 1 & -1/3 & \tabularnewline
    		                   & c & 0 & 1 &  2/3 & \tabularnewline
        \multirow{2}*{3$^{rd}$} & b & 0 & 1 & -1/3 & \tabularnewline
        						& t & 0 & 1 & 2/3 & \tabularnewline
        \hline %----------------------------
        \end{tabular}
    \end{center}

    \begin{center}
        \begin{tabular}{c|c|c|c}
        \hline %----------------------------
        Force & Gauge bosons & Mass (GeV/$c^2$) & Electric charge \tabularnewline
        \hline %----------------------------
        \hline %----------------------------
        Electromagnetic & $\gamma$ & 0 & 0 \tabularnewline  
        \multirow{2}*{Weak} & $Z^0$ & 91.1876 $\pm$ 0.0021& 0 \tabularnewline
             & $W^{\pm}$ & 80.3980 $\pm$ 0.0250 & $\pm 1$  \tabularnewline 
        Strong & g (8 gluons) & 0 & 0 \tabularnewline
        \hline %----------------------------

        \end{tabular}
    \end{center}

    There are four fundamental interactions in the nature: the gravity, the weak interaction, the electromagnetic interaction and the strong interaction.
    Those interactions have different characteristics and interact via gauge bosons.
    
    The electromagnetic (E.M.) interaction acts on the electric charges. 
    The range of the interaction is infinite. 
    The mediator for the E.M. interaction is the photon $\gamma$.
    It has a spin-0 and no mass. 
    
    The weak interaction acts on all the fermions.
    The mediators are the Z and W$^\pm$ bosons.
    The range of the interaction is only courte portée.

    The weak and E.M. can be combined together: electro-weak interaction.

    The strong interaction behaves on color-charges particles (quarks and gluons).
    They are 8 gauge bosons called gluon.
    The range is infinite due to the color charged and the confinement inside hadrons.

    The last force is the gravitation. 
    It is felt by all particles.
    The mediator is the graviton but is not yet observed.
    This interaction is not described by the S.M. and several theories try to explain that interaction.

   % The Standard Model describes very well three forces out of four: the electromagnetic (E.M.) interaction, the weak interaction and the strong.
   % The theory that explain the E.M. interaction is the Quantum Electrodynamics (QED). The vector of the interaction is the photon $\gamma$.
   % The weak interaction is the only one that takes part on all particles. 
   % It intermediates via Z and W$^\pm$ bosons.
   % The Quantum Chromodynamics (QCD) describes the strong interaction. 
   % The mediators of this interaction are the gluons to the number of eight. 
   % They can interact each other as well.

    \subsection{Interactions in the Standard Model}

	The mathematical basis of the S.M. is the Quantum Field Theory (QFT). All the interactions are described by the gauge group 
    \begin{equation}
    	SU_C(3) \otimes SU_L(2) \otimes U_Y(1)
	\end{equation}
    
    The gauge theory is invariant under a continuous set of local transformation.
    Taking the gauge symmetries and the least action into account, physicists were able to set up equations that describe the dynamic of the interactions by Lagrangian.
    
      \subsubsection{Quantum Electrondynamic}
	The Quantum Electrodynamic (QED) is used to described the electromagnetic interaction using a $U(1)$ gauge group. It is associating a Lagrangian for a fermion $\Psi$ with a massless photon field $A_\mu$:
    
    \begin{equation}
    	\mathcal{L}_{QED} =  \overline{\Psi}\left(x\right)\left( i \gamma^\mu D_\mu - m \right) \Psi\left(x\right) - \frac{1}{4}F_{\mu \nu}\left(x\right) F^{\mu \nu}\left(x\right)
    \end{equation}

	where $D_\mu \ = \ \partial_\mu - i Q_e A_\mu$ and $F_{\mu \nu} \ = \ \partial_\mu A_\nu - \partial_\nu A_\mu$.

    This Lagrangian is invariant under the transformation:

    \begin{equation}
        D_{\mu} \rightarrow e^{i\alpha(x)}D_{\mu}\Psi
    \end{equation}
    
      \subsubsection{Quantum Chromodynamics}
    The Quantum Chromodynamics (QCD) is the quantum field theory of the strong interaction.
    In this model, the interaction is du to a SU(3) gauge group. 
    It produces 8 gauge fields called gluons.
    The spinners of this theory are the six quarks that form a triplet with respect to the gauge symmetry.

    The SU(3) gauge group is a group of 9 - 1 = 8 real parameters and so of 8 generators. 
    Those generators are the Gell-Mann matrices. 
    The normalised generators are defined by: 
    
    \begin{equation}
        T^a = \frac{1}{2}\lambda^a
    \end{equation}

    The structure constant $f^{abc}$ can be expressed as:

    \begin{equation}
        if^{abc} = 2 Tr([T^a,T^b]T^c)
    \end{equation}
     
   Some theories arguments and the results of experiments in high energy physics ask to introduce six spinor fields, the quarks.
   Each of them are considered as a triplet state with respect to the SU(3) group:

   \begin{equation}
        q_i = \begin{pmatrix}
                q_i^1 \\
                q_i^2 \\
                q_i^3 \\
              \end{pmatrix}
   \end{equation}
    
    where $q_i$ are the six quarks.
    These quarks can appeared in three different states, called color and that are names red, blue and green.

    The local gauge symmetry U(1) should be included into the SU(3) group.
    
    The gauge field $A_{\mu}$ can be introduced in the group:
    
    \begin{equation}
        A_{\mu} = g_S A^a_{\mu}\frac{\lambda^a}{2}
    \end{equation} 
    with a = 1,...,8 and corresponds to the 8 gluons.
    A mass term  $m_g A^{\mu}_a A^a_{\mu}$ would not be gauge invariant, that implies the gluons are massless.

    The material derivative is then:

    \begin{equation}
        D_{\mu} = \partial_{\mu} - i A_{\mu} = \partial_{\mu} - i g_S A^a_{\mu} \frac{\lambda^a}{2}
    \end{equation}

    The QED field $F_{\mu \nu}$ is not gauge invariant in QCD.
    Nevertheless an additional term to obtain gauge invariant field tensor can be introduced:
    
    \begin{equation}
        G^a_{\mu \nu} = \left( \partial_{\mu} A^a_{\nu} - \partial_{\nu} A^a_{\mu} \right) + g_S f^{abc} A^b_{\mu} A^c_{\nu}
    \end{equation} 

    Finally, the QCD Lagrangian is given by:

    \begin{equation}
        \mathcal{L} = \sum_{i=1}^6  \bar{q_i} \left(i \gamma^{\mu}D_{\mu} -m_i \right)q_i - \frac{1}{4} G_{\mu \nu}^{a} G_{a}^{\mu \nu}
    \end{equation}
    
    \subsubsection{Electroweak lagrangian}
   
    Lagrangian EW goes here: 

  \section{The Higgs physics}

	One of the S.M. strength is to be able to explain the generation of the particle masses via the Higgs mechanism.
    
    \subsection{Symmetry Breaking}

    \subsection{Higgs mechanism}

  \section{Basic physics analysis at the ILC}


