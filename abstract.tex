\section*{Abstract}
%\addstarredchapter{Abstract}

The PLUME project develops ultra-light pixelated ladders following the requirements of the ILD vertex detector for the ILC. 
The double-sided ladders developed are made of 12 CMOS sensors positioned on each side of a mechanical support.
This work introduces a analysis of the $H\nu\nu$ final state at the centre-of-mass energy $350~\rm{GeV}$ and a luminosity of $250~\rm{fb}^{-1}$, where the Higgs boson is decaying into pair of quarks or gluons.
By applying consecutive cuts, it has been possible to enhance the signal contribution to study the Higgs boson.
This work includes also a study of the impact of mechanical deformations that degrade ladder performance.
A method to take into account these deformations during the off-line analysis is discussed.
Besides, a measurement of the radiation length of this prototype has been performed at DESY test beam.
First results give a material budget of $0.47~\pm~0.02~\%$ of $\rm{X_0}$.
Modules with a more constraint material budget have been tested in the laboratory.
The test done have shown that their performance is not degraded.


%Moreover, a study of the first prototype is introduced.

%On the instrumentation side, basic assessments performed in the laboratory for new ladder prototype with a material budget of $0.35~\%$ of $\rm{X_0}$ are presented.

%Basic assessments performed in the lab for new ladder version with a material budget of $0.35~\%$ of X0 is presented.
%Impact of mechanical deformations on ladder performance is studied.
%An algorithm to take into account these mechanical deformations during off-line analysis is discussed.
%The last point of the thesis is focused on a protocole to measure the material budget thanks to the beam delivered at DESY.
 
\paragraph{Keywords:} ILC, ILD, CMOS sensor, PLUME ladder, material budget, radiation length, spatial resolution, mechanical deformation

\newpage

\section*{Résumé}

Le projet PLUME développe des échelles ultra-légères en suivant le cahier des charges du détecteur de vertex pour l'ILC.
Les échelles de détection double-face développées sont constituées de 12 capteurs CMOS repartis sur chaque face d’un support mécanique.
Ce travail introduit une analyse du canal menant à l’état final $H\nu\nu$ pour une énergie de $350~\rm{GeV}$ et une luminosité de $250~\rm{fb}^{-1}$, où le boson de Higgs se désintègre en pair de quarks ou de gluons. 
En appliquant des coupures consécutives, il a été possible d'améliorer la contribution du signal afin de pouvoir étudier le boson de Higgs.
Ce travail contient aussi une étude de l'impact des éformations mécaniques qui dégradent les performances de l'échelle.
Une méthode pour prendre en compte ces déformations pendant l'analyse hors-ligne est discutée.
De plus, une mesure de la longueur de radiation de ce prototype a été réalisée au DESY.
Les premiers résultats donnent un budget de matière de $0.47 \pm 0.02~\%$ de $\rm{X_0}$.
Des modules avec un budget de matière plus contraint ont été testé en laboratoire.
Les tests effectués ont montré que leurs performances ne sont pas dégradés. 

\paragraph{Mots-clés :} ILC, ILD, capteur CMOS, échelle PLUME, budget de matière, longueur de radiation, résolution spatiale, déformation mécanique
