\section*{Abstract}
%\addstarredchapter{Abstract}

PLUME collaboration develops prototype of vertex detector for the ILC. 
The double-sided ladders developed are made of 12 CMOS sensors positioned on each side of a mechanical support.
This work presents a analysis of the Hvv final state at the centre-of-mass energy 350 GeV and a luminosity of 250 fb-1, where the Higgs boson is decaying into pair of quarks or gluons.
Basic assessments performed in the lab for new ladder version with a material budget of 0.35 % of X0 is presented.
Impact of mechanical deformations on ladder performance is studied.
An algorithm to take into account these mechanical deformations during off-line analysis is discussed.
The last point of the thesis is focused on a protocole to measure the material budget thanks to the beam delivered at DESY.
 
\paragraph{Keywords:} ILC, ILD, CMOS sensor, PLUME ladder, material budget, radiation length, spatial resolution, mechanical deformation

\newpage

\section*{Résumé}

La collaboration PLUME développe des prototypes de détecteur de vertex pour l’ILC.  
Les échelles de détection double-face développées sont constituées de 12 capteurs CMOS repartis sur chaque face d’un support mécanique.
Ce travail présente une analyse du canal menant à l’état final Hvv pour une énergie de 350 GeV et une luminosité de 250 fb-1, où le boson de Higgs se désintègre en pair de quarks ou de gluons. 
Les évaluations basiques effectuées en laboratoire pour les nouvelles versions d’échelles avec un budget de matière de 0.35% de X0 sont présentées.
L’impact des déformations mécaniques sur les performances de l’échelle est étudié et un algorithme permettant de les prendre en compte pendant l’analyse hors-ligne est discutée.
Le dernier point de la thèse s’intéresse à un protocole de mesure du budget de matière grâce au faisceau délivré par DESY.

\paragraph{Mots-clés :} ILC, ILD, capteur CMOS, échelle PLUME, budget de matière, longueur de radiation, résolution spatiale, déformation mécanique
