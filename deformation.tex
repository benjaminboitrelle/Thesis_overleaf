\chapter{Deformation studies of a ladder under the test beam}

  The first full-scale prototype which embeds twelve sensors glued on a copper flex-cable and a 8 \% density \gls{SiC} foam was tested in November 2011 at CERN-SPS facility with a pions beam of 120 GeV.
  This chapter does not aim to present the test beam campaign and the results, but focus on a specific study for the ladder titled in one direction with respect to the beam axis.
  
  \minitoc

  \section{PLUME-V1 tested at CERN-SPS}
    \subsection{Experimental set-up}

     \begin{itemize}
       \item Beam structure
       \item Telescopes
     \end{itemize}

    \begin{figure}
      %\centering
      \missingfigure{TB geometries}
    \end{figure}

    \subsection{Software analysis}

      TAF:
      \begin{itemize}
        \item ROOT/C++
      \end{itemize}

  \section{Deformation studies}
    
    This test-beam results have already been discussed (reference to paper and thesis) and results, such as the efficiency or ... are no going to be presented.

    \begin{figure}
      %\centering
      \missingfigure{Track-hit residual as a function of the hit position (normal incidence)}
    \end{figure}

    \begin{figure}
      %\centering
      \missingfigure{Track-hit residual as a function of the hit position (tilt)}
    \end{figure}

    \begin{figure}
      %\centering
      \missingfigure{Residual (tilt and normal incidence)}
    \end{figure}

    \begin{figure}
      %\centering
      \missingfigure{Explanation of deviations}
    \end{figure}
    
  \section{Benefits of double-sided measurement}

  %\todo{REF Loic thesis for TB@CERN results}
