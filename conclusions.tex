%\chapter*{Conclusions}
%\chapter*{Conclusions and outlooks\markboth{CONCLUSIONS AND OUTLOOKS}{CONCLUSIONS AND OUTLOOKS}} \addcontentsline{toc}{chapter}{\protect\numberline{}Conclusions and outlooks}
\chapter*{Conclusions and outlooks\markboth{CONCLUSIONS AND OUTLOOK}{CONCLUSIONS AND OUTLOOK}}
\addstarredchapter{Conclusions and outlook}

Over the last few years, the scientific community agrees that an accelerator beyond the \gls{LHC} is needed.
This next generation accelerator is required for high-precision measurements, specifically for characterising the electroweak symmetry breaking, as well as looking for events compatible with physics beyond the \gls{SM}.
So far, the most advanced candidate is the \gls{ILC}.
With the technical design report~\cite{Baer2013}, physicists have confirmed that the community is ready to build this future linear collider.
A governmental decision is eagerly awaited to start the project.
The work performed during this thesis was specifically focused on the feasibility to construct double-sided pixelated ladders suited for a vertex detector at the \gls{ILC}, that features a low material budget (below $0.35~\%~\rm{X_{0}}$) and a spatial resolution below $ 3~\rm{\mu m}$ for the \gls{ILD}.

This work is introduced with a physics analysis of simulated data corresponding to collisions in the \gls{ILC} at a centre-of-mass energy $\sqrt{s} = 350~\rm{GeV}$.
The channel studied underlines the benefits of event types produced by $e^{+}e^{-}$ collisions.
It is the one leading to a $\nu \overline{\nu} H$ final state, with the Higgs boson decaying into a pair of quarks or gluons.

The large part of the final particles produced in such events are invisible to the detector. 
Hence a corner stone of the analysis of this channel is the ability to drastically reduce the contribution of events leading to the same detector response or a similar event signature. 
The result presented in this work shows that applying a sequence of cuts on eight identified discriminating variables allows to reach a significance of 64 for the signal within the selected events. 
This level is however not sufficient for the precision measurement targeted with these events, which requires a significance of at least 70. 
Though sequential cuts are able to reduce the background events by three orders of magnitude, they also remove half of the signal. 
To improve on result, another thesis conducted by Felix \textsc{Mueller}~\cite{Mueller} developed a multivariate analysis  approach, which finally reached the desired significance for the event selection process.

After this first important analysis step, the actual characterization of the Higgs boson can start. 
But this second study was beyond the scope of this work even if it is strongly related to the goal. 
Indeed this characterization requires among others to identify jets issued from $c$ quark against those produced by $b$ quarks. 
This ability sets stringent specifications on the vertex detector layers, since the identification relies on the short lifetime difference between these two types of quarks. 
The \gls{PLUME} collaboration hopes to demonstrate the feasibility of such layers, exploiting the concept of double-sided ladders.


%To reduce the contribution of events leading to the same detector response or a similar event signature, several hard cuts depending on different parameters are applied. 
%After eight cuts, the background events are three orders smaller but this method implies a lose of half of the signal.
%Due to the restricted time to conduct this work, a different selection technique could not be used.
%The work performed by Felix \textsc{Mueller} has nevertheless, shown that a TMVA solution will keep the signal significant while reducing the background contribution~\cite{Mueller}.
%This analysis has to be completed with a study of the Higgs boson decaying into a pair of charmed and anti-charmed quarks.
%The ability of measuring the coupling of the Higgs boson to the $c$ quarks is needed. 
%This measurement is challenging to perform at the \gls{LHC} due to the \gls{QCD} background. 
%Moreover, the life-time of the $b$ and $c$ quarks are similar, leading to a close vertex decay. 
%Thus, the vertex detector has to be optimised to separate this two quarks.

The largest part of this thesis work was devoted to study \gls{PLUME} prototypes, which are currently equipped with \gls{CMOS} pixel sensors.
Since the first small scale prototype (V0) developed in 2009, the collaboration has shown its capability to build full scale and fully functional \gls{PLUME} ladders.
Two versions have been produced so far, the first one (V1) with a relaxed material budget constraint and the final one approaching the $0.3~\%$ of $\rm{X_{0}}$ requirement.
The work done during these three years was split into checking the basic assessments of individual \gls{PLUME} module before assembling them into ladders, studying the impact of the mechanical deformations on the pointing resolution and preparing a protocol to measure the radiation length of complete ladders.

The basic assessments are done in the laboratory.
For each module, an optical survey is performed to ensure the right positioning of the sensors.
Moreover, chips and wire-bonds are checked to make sure that none of them were damaged during gluing procedure or transportation.
Then, an electrical test is performed and the six sensors are validated and characterised.
The results of these tests have shown that the combination of the six sensors running at the same time in close proximity does not degrade the \gls{MIMOSA}-26 performance.
The fake hit rate of each sensor measured for the different modules is below $10^{-6}~\rm{hits/pixel/events}$ at a threshold of $6~\sigma$.

During the data analysis of the test beam performed in 2011 at \gls{CERN} with a V1 ladder, runs where the ladder was tilted with respect to the track direction have shown deviation of the spatial resolution, as well as a correlation between the position of the hit on the sensor and the measured track residual.
These deviations come from mechanical deformation induced by the materials used (flex-cables, foam) and the mounting procedure of a \gls{PLUME} ladder.
An offline algorithm to reduce the impact of these deformations has been implemented and has shown a good improvement on the spatial resolution measured for titled tracks.
A second study during this data analysis has permitted one to present the benefits of a double-sided measurement. 
The spatial resolution has been improved by a factor of about $1/\sqrt{2}$ and the creation of mini-vectors gives access to a new parameter, the angular resolution, which was measured to be $0.1^{\circ}$ at normal incidence.

Finally, the last work performed during this thesis was to set-up a test beam with a lower beam energy (up to $5~\rm{GeV}$ electrons).
One of the goals is to measure the radiation length of the V1 ladder.
After analysing the data, the radiation length measured is $\left. \frac{x}{X_0} \right|_{\rm{measured}} \simeq 0.47 \pm 0.02~\%~\rm{X_0}$, confirming the theoretical calculation of $\left. \frac{x}{X_0}\right|_{\rm{theoretical}} \simeq 0.498~~\%~\rm{X_0}$.
To improve the precision on the measurement, different methods could be used.
Instead of considering the \gls{PLUME} sensors as reference planes, the six telescope planes could be used for tracking and measuring the radiation length.
Another solution could be to perform a calibration run with well-known materials. 
The reconstructed angle would be then corrected by a factor determined during the calibration procedure.


% OUTLOOKS:
%  * Test beam at low energy
%    - Efficiency and resolution?
%  * Power-pulsing
%    - Mi-26?
%    - Magnetic field?
%  * New structure
%    - Sernwiette? 

The work performed during this thesis has shown that the collaboration is able to build a lightweight mechanical structure for a vertex detector.
A way to overcome the mechanical deformation inherent to the thin-sensor concept has been discussed.
A procedure to control the material budget has been implemented and the first results are encouraging.
This thesis, as well as the work performed by Loic \textsc{Cousin}~\cite{cousin} and Robert \textsc{Maria}~\cite{maria}, has shown that \gls{CMOS} sensors are not disturbed when they are closely laid together (butted and facing each other).
The expected performance for one sensor is preserved for a ladder.
Nevertheless, more tests and optimisations have to be done to improve the ladder.

Firstly, the results of the test beam performed in April 2016 presented here are focused only on the material budget measurement.
The ladder performances (efficiency and spatial resolution) for low momentum particles have to be checked.
Runs with different tilts and air flow speeds were acquired in order to study the mechanical deformations in more detail.

Secondly, the ladder new prototype with a material budget of $0.35~\%$ of $\rm{X_{0}}$ has been tested only in the laboratory.
A test beam in real conditions to measure its performance and its material budget has to be planned.
Depending on the results of the test beam, the collaboration could consider achieving an even lower material budget.
Another \gls{SiC} foam with a density of $2~\%$ or a different bonding technique could be used.
Two new bonding  techniques could be considered: laser soldering or embedding the sensors directly inside the multi-layer micro-cable~\cite{Baudot2012}.
This second technique consists of gluing the chips on a polymide substrate layer.
Then, a metal layer is deposited on top of it and the metal traces are directly connected to the chips pads.
On the last step, an insulator is added to the module.
These methods offer the advantage of avoiding wire-bondings and reducing the width of the module.
Moreover, with this structure, mechanical stress is applied on the polymer wrapping and it reduces its impact on the sensors.

Thirdly, the lightweight mechanical structure is validated with \gls{MIMOSA}-26 sensors.
They have the advantage of having a continuous readout without suffering from dead time, but their integration time is rather slow ($115.2~\rm{\mu s}$).
This does not allow for tagging tracks with the bunch crossing.
However, to keep the material budget below $1.5~\%~\rm{X_{0}}$, the power consumption of sensors has to be adapted to the cooling system used.
One way for decreasing the power consumption is to use a \textit{power-pulsing} scheme.
The sensor's consumption is reduced during the $200~\rm{ms}$ of dead time and increased again before the next bunch crossing.
The power-pulsing scheme has been tested on a single \gls{MIMOSA}-26 sensor, but not yet on a complete ladder.
The results have shown that during the inactive period, the nominal supply voltage can de decreased from $3.3~\rm{V}$ to $1.85~\rm{V}$ without losing the sensor's registers~\cite{Kuprash2013}.
Nonetheless, this sensor is not designed for this purpose and its behavior with other sensors inside a complete ladder is not yet known.
For a complete ladder, basic assessments with power-pulsing should be done in the laboratory, before performing tests in real conditions with a high magnetic field (more than $1~\rm{T}$).
The impact of the Lorentz forces on a $10~\rm{g}$ ladder need to be studied, specifically to look for unwanted deformations or vibrations induced on the mechanical structure.

Finally, the double-sided concept could be enriched if sensors with different optimisations are mixed on both sides.
One side would have highly granular sensors providing a good spatial resolution (below $3~\rm{\mu m}$), whereas the other one would have sensors with elongated pixel providing a fast integration time (well below $10~\rm{\mu s}$).
Basic assessments, as well as the power dissipation have to be studied for this configuration.
If the power dissipation is too high, the air-cooling system might not be performant anymore.
Microchannels could be integrated inside the mechanical structure, in which a coolant will be in charge of regulating the power dissipation.

%Finally, the mechanical structure can be improved by embedding the sensors directly inside the multi-lay micro-cable~\cite{Baudot2012}.
%This technique consists to glue the chips on a first polymide substrate layer.
%Then, a metal layer is deposited on top of it and the metal traces are directly connected to the chips pads.
%On the last step, an insulator is added to the module.
%This method offers the advantage of avoiding the wire-bonding and reducing in the same time the width of the module.
%Moreover, with this structure, mechanical stress is applied on the polymer wrapping and reducing its impact on the sensors.
%The advantages of this technique are the ability to connect directly the metal traces to the pads, avoiding wire-bonding and reducing in the same time the width of the module.

