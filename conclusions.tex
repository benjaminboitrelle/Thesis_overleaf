%\chapter*{Conclusions}
\chapter*{Conclusions and outlooks\markboth{CONCLUSIONS AND OUTLOOKS}{CONCLUSIONS AND OUTLOOKS}} \addcontentsline{toc}{chapter}{\protect\numberline{}Conclusions and outlooks}

Over the last years, the scientific community agrees that an accelerator beyond the \gls{LHC} is needed.
This next generation accelerator is required for high-precision measurements, specifically for characterising the electroweak symmetry breaking, as well as looking for event compatible with physics beyond the \gls{SM}.
So far, the most advanced candidate is the \gls{ILC}.
With the technical design report~\cite{Baer2013}, the physicists have confirmed that the community is ready to build this future linear collider.
A governmental decision is eagerly awaited to start the project.
The work performed during this thesis was specifically focused on the feasibility to construct double-sided pixelated layers suited for a vertex detector at the \gls{ILC}, that features a low material budget (below $0.35~\%~\rm{X_{0}}$) and a spatial resolution below $\sim 3~\rm{\mu m}$ for the \gls{ILD}.

This work is introduced with a physics analysis on simulated data corresponding to collision in the \gls{ILC} at a centre-of-mass energy $\sqrt{s} = 350~\rm{GeV}$.
The channel studied underlines the benefits of event types produced by $e^{+}e^{-}$ collisions, it is the one leading to a $\nu \overline{\nu} H$ final state, with the Higgs boson decaying into pair of quarks or gluons.

The large part of the final particles produced in such events are invisible to the detector. 
Hence a corner stone of the analysis of this channel is the ability to drastically reduce the contribution of events leading to the same detector response or a similar event signature. 
The result presented in this work shows that applying a sequence of cuts on eight identified discriminating variables allows to reach a significance of 64 for the signal within the selected events. 
This level is however not sufficient for the precision measurement targeted with these events, which requires a significance of 70. 
Though sequential cuts are able to reduce by three orders of magnitude the background events, in the meantime they also remove half of the signal onces. 
To improve on our result, another thesis work conducted by Felix \textsc{Mueller}~\cite{Mueller} developed a multivariate analysis  approach, which finally reached the desired significance for the event selection process.

After this first important analysis step, actual characterization of the Higgs boson can start. 
But this second study was beyond the scope of this work even if it is strongly related to our goal. 
Indeed this characterization requires among others to identify jets issued from $c$ quark against those produced by $b$ quarks. 
This ability sets stringent specifications on the vertex detector layers, since the identification relies on the short lifetime difference between these two types of quarks. 
The \gls{PLUME} collaboration ambitions to demonstrate the feasibility of such layers, exploiting the concept of double-sided ladders.


%To reduce the contribution of events leading to the same detector response or a similar event signature, several hard cuts depending on different parameters are applied. 
%After eight cuts, the background events are three orders smaller but this method implies a lose of half of the signal.
%Due to the restricted time to conduct this work, a different selection technique could not be used.
%The work performed by Felix \textsc{Mueller} has nevertheless, shown that a TMVA solution will keep the signal significant while reducing the background contribution~\cite{Mueller}.
%This analysis has to be completed with a study of the Higgs boson decaying into a pair of charmed and anti-charmed quarks.
%The ability of measuring the coupling of the Higgs boson to the $c$ quarks is needed. 
%This measurement is challenging to perform at the \gls{LHC} due to the \gls{QCD} background. 
%Moreover, the life-time of the $b$ and $c$ quarks are similar, leading to a close vertex decay. 
%Thus, the vertex detector has to be optimised to separate this two quarks.

The largest part of this thesis work was devoted to study \gls{PLUME} prototypes, which are currently equipped with \gls{CMOS} pixel sensors.
Since the first small scale prototype (V0) developed in 2009, the collaboration has shown its capability to build full scale and fully functional \gls{PLUME} ladders.
Two versions have been produced so far, the first one (V1) with a relaxed material budget constraint and the final one approaching the $0.3~\%$ of $\rm{X_{0}}$ requirement.
The work done during this three years was split into checking the basic assessments of individual \gls{PLUME} module before assembling them into ladders, studying the impact of the mechanical deformations on the pointing resolution and preparing a protocol to measure the radiation length of complete ladders.

The validation and characterisation of the final ladder done in laboratory have shown that the combination of six sensors running at the same time does not degrade the performances of the \gls{MIMOSA}-26.
The fake hit rate measured for the different modules is below 

During the data analysis of the test beam performed in 2011 at \gls{CERN} with the V1 ladder, runs where the ladder was tilted with respect to the track direction have shown deviation on the pointing resolution, as well as a correlation between the position of the hit on the sensor and the track residual measured.
These deviations are coming from mechanical deformation induced by the material used (flex-cable, foam) during the mounting procedure of a \gls{PLUME} ladder.
An algorithm to reduce the impact of these deformations have been implemented and has shown a good improvement on the pointing resolution measured for titled tracks.
A second study during this data analysis has permitted to present the benefits of a double-sided measurement. 
The spatial resolution has been improve by a factor of about $1/\sqrt{2}$ and the creation of mini-vectors give access to a new parameter, the angular resolution, which was measured to be $0.1^{\circ}$ at normal incidence.

Finally, the last work performed during the thesis was to set-up a test beam with a lower beam energy (up to $5~\rm{GeV}$ electrons) to measure the radiation length of the first fully functional prototype.
After analysing the data, the radiation length measured is $\left. \frac{x}{X_0} \right|_{\rm{measured}} \simeq 0.47 \pm 0.02~\%~\rm{X_0}$, confirming the theoretical calculation of $\left. \frac{x}{X_0}\right|_{\rm{theoretical}} \simeq 0.498~~\%~\rm{X_0}$.
This measurement is in a good agreement to the theoretical value determined, even if the measured value is a bit lower than the expectations. 
To improve the precision on the measurements, different methods could be used.
Instead of using the \gls{PLUME} sensors as reference planes, the six telescope planes could be used to measured the deflection of the complete ladder.

% OUTLOOKS:
%  * Test beam at low energy
%    - Efficiency and resolution?
%  * Power-pulsing
%    - Mi-26?
%    - Magnetic field?
%  * New structure
%    - Sernwiette? 

The work performed during this thesis has shown that the collaboration is able to build lightweight mechanical structure for a vertex detector.
The material budget has been controlled and the first results are encouraging, a way to overcome the mechanical deformation inherent to the thin-sensor concept have been discussed.
Moreover, this thesis, as well as the work performed by Loic \textit{Cousin} and Robert \textit{Maria}, have shown that \gls{CMOS} sensors are disturbed when they are closely laid together (butted and facing each others), but the expected performances (efficiency and spatial resolution) for one sensor are measured for a ladder.
Nevertheless, the new ladder prototype with a material budget of $0.35~\%$ of $\rm{X_{0}}$ has not been tested in real conditions yet.



Nevertheless, more tests and optimisations have to be done to improve the ladder.

Firstly, the results of the test beam performed in April 2016 presented here are focused only on the material budget measurement.
The performances of the ladder (efficiency and spatial resolution) for low momentum particles have to be checked.
Runs with different tilts and air flow speed were acquired in order to study the mechanical deformations in more details.

Secondly, the sensors used for validating the mechanical structure are \gls{MIMOSA}-26.
They have the advantage of having a continuous readout without suffering from dead time, but their integration time is rather slow ($115.2~\rm{\mu s}$).
This imposes to readout the data between two bunch crossings at the \gls{ILC}.
But to keep the material budget below $1.5~\%~\rm{X_{0}}$, the power consumption of the sensors have to be adapted to the cooling system used.
One way to decrease the power consumption is to use a \textit{power-pulsing} solution.
This consists to reduce the consumption of the sensors during the $200~\rm{ms}$ of dead time and switching them on before the bunch crossing.
Tests to perform \textit{power-pulsing} on a single \gls{MIMOSA}-26 sensor have been performed by the collaboration.
The results have shown that the nominal supply voltage can de decreased from $3.3~\rm{V}$ to $1.85~\rm{V}$ without losing the sensor's registers~\cite{Kuprash2013}.
Nonetheless, this sensor is not designed for this purpose and its behavior with other sensors inside a complete ladder is not known yet.
Different tests have to be done to make sure that the ladder is still behaving correctly in a \textit{power-pulsing} mode.
If the results are comforting, the ladder have to be then tested under real conditions with a high magnetic field (ideally $3~\rm{T}$).
This tests is needed to study the impact of the Lorentz forces on the whole ladder, especially to know if it induces unwanted deformations or vibrations of the ladder.

Finally, the mechanical structure can be improved by embedding the sensors directly inside the multi-lay micro-cable~\cite{Baudot2012}.
This technique consists to glue the chips on a first polymide substrate layer.
Then, a metal layer is deposited on top of it and the metal traces are directly connected to the chips pads.
On the last step, an insulator is added to the module.
This method offers the advantage of avoiding the wire-bonding and reducing in the same time the width of the module.
Moreover, with this structure, mechanical stress is applied on the polymer wrapping and reducing its impact on the sensors.
The advantages of this technique are the ability to connect directly the metal traces to the pads, avoiding wire-bonding and reducing in the same time the width of the module.

