\chapter{Test of a PLUME-V1 at low energy with the DESY test beam}

\minitoc

  \section{PLUME-V1 tested at DESY}

   A second test beam was performed in April 2016 at DESY with positrons up to 5 GeV. 
   The goal of this test beam was to study the performance of this device with low-momentum particles.
   The ladder tested as a version-1, but not the one already tested at CERN.
   The steps, from the preparation to the analysis are explained here.

    \subsection{Test beam preparation}

    The DATURA telescope was used in TB-21. 
    \begin{itemize}
      \item EUDAQ and check acquisition stability
      \item Estimation of spatial resolution for different geometries
      \item Rotation stage
    \end{itemize}

    \begin{figure}
      %\centering
      \missingfigure{Plots to estimate spatial resolution with different geometries}
    \end{figure}

    \begin{figure}
      %\centering
      \missingfigure{Mechanics}
    \end{figure}

    \subsection{experimental set-up}

    \begin{itemize}
      \item beam structure
      \item telescope
      \item Fan
    \end{itemize}
    \begin{figure}
      %\centering
      \missingfigure{Picture of telescope and PLUME}
    \end{figure}

    \subsection{Software analysis}

    Due to the specific acquisition which was done with EUDAQ, the analysis was a bit complicated.
    EUTelescope was coded to expect six telescope planes plus a single DUT.
    One way to overcome this problem was to perform a biased analysis.
    Instead of performing the alignment of the telescopes and then the DUT for an analysis, this step has to be done in one way.
    Nevertheless, the alignment procedure was not working.

    To perform the alignment, I have used a python script written by Claus Kleinwort, which reads the hit position of every sensor and use GBL and MP-II to perform the alignment.

