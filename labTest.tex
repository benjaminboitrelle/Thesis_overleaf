\chapter{Electrical Validation and laboratory testing}
\label{chap:labTests}


  This chapter is introducing the steps to characterise and validate a sensor in lab before testing it in real conditions.

  \begin{itemize}
    \item Describe bench
    \item Describe control of a sensor
    \item Describe output of Mi-26
  \end{itemize}

\section{Mechanical check}
  \begin{itemize}
    \item Check positioning matrix defect (crack)
    \item Check wire bounds
    \item Check alignment
  \end{itemize}

  \begin{figure}
    \missingfigure{Alignment of two sensors}
    \caption{Alignment check made with a microscope.}
  \end{figure}

  \begin{figure}
    \missingfigure{Wire bounds crashed}
    \caption{One has to be careful!}
  \end{figure}

\section{Electrical validation}

  \begin{itemize}
    \item Check consumption
    \item Check JTAG control
    \item Check Output
  \end{itemize}

\section{Noise estimation}

  \begin{itemize} 
    \item Describe steps
    \item Describe analysis
    \item FHR with normal acquisition
  \end{itemize}

  \begin{figure}
    \missingfigure{Sub-matrix response A.K S-curve}
    \caption{Sub-matrix response}
  \end{figure}

  \begin{figure}
    \missingfigure{TN and FPN}
    \caption{TN and FPN}
  \end{figure}

  \begin{figure}
    \missingfigure{Output Fe55}
    \caption{Output of 10k events acquisition with a Fe55 radiation source.}
  \end{figure}

\section{Cluster}
